\documentclass[11pt,a4paper]{report}
\usepackage[utf8]{inputenc}
\usepackage{amsmath}
\usepackage{amsfonts}
\usepackage{amssymb}
\usepackage{graphicx}
\usepackage[left=1.00cm, right=1.00cm, top=1.00cm, bottom=2.00cm]{geometry}
\author{Stéphane FEUGA}
\title{{\huge Rapport de Stage - Développeur Logiciel} \\
{\normalsize Stage du 02/04/2014 au 31/05/2014 pour l'association Uncanny}}
\begin{document}

\maketitle

\vspace*{\stretch{1}}
\begin{center}
This page intentionally left blank
\thispagestyle{empty}
\end{center}
\vspace*{\stretch{1}}

\tableofcontents

\chapter{Introduction}
	...---...
	
\chapter{Remerciements}
	\section*{}
		Tout d'abord, je souhaite  remercier Mr Cédric CHERDEL et Mr Laurent CEBE de l'association Uncanny qui m'ont offert la possibilité d'avoir un sujet de stage en adéquation avec mes capacités et les compétences à mettre en œuvre pour la validation de ce stage. \\
		Je souhaite aussi remercier Gabriel BLOCK, Erwan FOURNEL, Emanuelle FERRAND et Florence NATIVELLE (tous de l'I.M.I.E.) pour leurs implications dans cette formation.
	\section*{}
		Je souhaite enfin remercier ma femme Akané ainsi qu'Anaël Tessier.

\chapter{Compétences à mettre en œuvre}
	\section*{Compétence Obligatoire}
		\begin{quote}
		Développer une interface utilisateur.
		\end{quote}
	\section*{Compétence Choisie}
		\begin{quote}
		Mettre en œuvre une solution de gestion de contenu ou
		d’e-commerce.
		\end{quote}

\chapter{L'Association Uncanny}
	\section{Historique}
		...---...
	\section{Présentation du projet}
		Le projet est la mise en place de deux sites internet pour promouvoir les projets et activités de l'association Uncanny. 
		\subsection{Objectifs}
			\begin{enumerate}
				\item Une page d'accueil animé donnant accès aux contenues.\\
				Cette page sera en HTML5 et responsive.
				\item Création d'un theme responsive en HTML5 pour Wordpress.
				\item Un site internet sur la production de danse contemporaine.\\
				Mise en place d'un Wordpress.
				\item Un site lié à l'activité de massage THAÏ.\\
				Mise en place d'un Wordpress.
			\end{enumerate}
		\subsection{Cible}
			\begin{itemize}
				\item Les professionnels de la Danse
				\item Les particuliers
				\item Les compagnies de Danse
				\item Les Mairies
				\item Les Départements et Régions
			\end{itemize}
		\subsection{Les acteurs}
			\begin{itemize}
				\item Cédric CHERDEL
				\item Laurent CEBE
			\end{itemize}
		\subsection{L'existant}
			\begin{enumerate}
				\item Une charte Graphique
				\item Une liste de contact
				\item Des documentations sur les projets de danse
			\end{enumerate}
\chapter{Le Stage}
	\section{Présentation du projet Danse}
		\subsection{Le contexte}
		\subsection{La problématique}
	\section{Présentation du projet Massage}
		\subsection{Le contexte}
		\subsection{La problématique}
	\section{Détail des fonctionnalités}
		\subsection{Danse}
			\paragraph{Fonctionnalités}
			\paragraph{Contrainte}
		\subsection{Massage}
			\paragraph{Fonctionnalités}
			\paragraph{Contrainte}
	\section{L'Expérience Utilisateur et l'Accessibilité}
		\subsection{Danse \& Massages}
			\paragraph{Le Maquettage}
				\subparagraph{Zoning}
					...---... Ici mon Zoning
				\subparagraph{Wireframe}
					...---... Ici mon Wireframe
				\subparagraph{Prototype}
					...---... Ici mon screenshot du prototype
				\subparagraph{Mood Board, Styles Tiles \& Mockup}
					...---... Ici le screenshot des éléments que l'on m'as fournis (psd, ai et pdf)
	\section{Planning: méthodes et outils}
		\subsection{Cahier des charges}
			J'ai commencé par rédiger un premier cahier des charge vierge (basé sur un modèle fournie par l'I.M.I.E.). Lors du premier jours, j'ai expliqué rapidement l'utilité d'un cahier des charge, puis dans une deuxièmes réunion, nous avons échangés autour des questions destiné à remplir ce cahier des charge. Enfin, lors de la réunion de validation, nous avons repris tout les points afin de les valider.
		\subsection{WBS}
		\subsection{Pert \& Gantt}
		\subsection{Agilité \& Kanban Board}

\chapter{Conception}
	\section{Architecture}
        \subsection{Wordpress}
	\section{Modélisations}
		\subsection{UML}
			\paragraph{Use Case}
			\paragraph{Deployment}
	\section{Test \& Qualité}
		\subsection{Unitaire}
		\subsection{Fonctionnel}
		\subsection{Intégration}

\chapter{Développement}
	\section{Technologies}
	\section{Danse \& Massage}
		\subsection{Vue (Couche Présentation)}
			\paragraph{Pages HTML}
			\paragraph{CSS \& Responsive Design}
			\paragraph{Javascript}
		\subsection{Contrôleur (Couche Métier)}
			\paragraph{API REST}
			\paragraph{JSON}
			\paragraph{Ajout de média (Photos, Vidéo et PDF)}
			\paragraph{Authentification}
			\paragraph{Gestion de calendrier (et des Rendez-vous)}
			\paragraph{Envoi de message}
	\section{Réalisation des tests d'intégration}

\chapter{Déploiement \& Recette}
	\section{Prérequis techniques}
	\section{Scripts de mise en place de l'architecture logiciels}
	\section{Scripts de déploiement de l'application}
	\section{Documentation:}
		\subsection{Utilisateur}
		\subsection{Développeur}
		\subsection{Administrateur}
	\section{Recette}

\chapter{Bilan \& Conclusion}

\chapter*{Lexique}

\pagenumbering{Roman}
\listoffigures
\listoftables

\chapter*{Annexes}
\end{document}
